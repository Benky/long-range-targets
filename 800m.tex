% Copyright (c) 2013 Benky
% This program is made available under the terms of the MIT License.
\documentclass{article}
\usepackage{tikz}
\usepackage{mathpazo}
\usepackage[top=20mm,left=20mm,right=0mm,bottom=0mm,a0paper]{geometry}
\usepackage{verbatim}

\begin{document}

\newcommand{\smallFont}{9.0}
\newcommand{\largeFont}{15.0}
\newcommand{\circleDistance}{75mm} % distance between two circles (not valid for X-10 circles)
\newcommand{\circleTen}{88.5mm} % radius of the 10 circle
\newcommand{\baseOffset}{50mm} % some random number :)
\newcommand{\xLength}{15mm} % half of the length of the X line
\newcommand{\textTen}{\baseOffset/2+\circleTen/2}
\newcommand{\textTenNeg}{-\baseOffset/2-\circleTen/2}

\begin{tikzpicture}
  \begin{scope}[shift={(210mm,300mm)}]
    \draw[line width=2mm, fill=black] (0cm, 0cm) circle (\circleTen+\circleDistance); % 9
    \draw[line width=2mm, color=white, fill=black] (0cm, 0cm) circle (\circleTen); % 10
    \draw[line width=2mm, color=black, fill=white] (-\baseOffset, -\baseOffset) rectangle (\baseOffset, \baseOffset); % box
    \draw[line width=2mm, color=black, fill=white] (0cm, 0cm) circle (\circleDistance/2); % X    

    \draw[line width=8mm] (-\xLength, -\xLength) -- (\xLength, \xLength);
    \draw[line width=8mm] (-\xLength,  \xLength) -- (\xLength, -\xLength);
    
    \node[scale=\smallFont, color=white] at (0mm, \textTen) {\large \textbf{10}};
    \node[scale=\largeFont, color=white] at (0mm, \baseOffset+\circleDistance) {\large \textbf{9}};
    \node[scale=\smallFont, color=white] at (0mm, \textTenNeg) {\large \textbf{10}};
    \node[scale=\largeFont, color=white] at (0mm, -\baseOffset-\circleDistance) {\large \textbf{9}};
    
    \draw[line width=2mm] (0cm, 0cm) circle (\circleTen+2*\circleDistance);
    \draw[line width=2mm] (0cm, 0cm) circle (\circleTen+3*\circleDistance);
    \draw[line width=2mm] (0cm, 0cm) circle (\circleTen+4*\circleDistance);

    \node[scale=\largeFont, color=black] at (\baseOffset+2*\circleDistance, 0mm) {\large \textbf{8}};
    \node[scale=\largeFont, color=black] at (\baseOffset+3*\circleDistance, 0mm) {\large \textbf{7}};
    \node[scale=\largeFont, color=black] at (\baseOffset+4*\circleDistance, 0mm) {\large \textbf{6}};        

    \node[scale=\largeFont, color=black] at (-\baseOffset-2*\circleDistance, 0mm) {\large \textbf{8}};
    \node[scale=\largeFont, color=black] at (-\baseOffset-3*\circleDistance, 0mm) {\large \textbf{7}};
    \node[scale=\largeFont, color=black] at (-\baseOffset-4*\circleDistance, 0mm) {\large \textbf{6}};        
\end{scope}
\end{tikzpicture}


\end{document}
